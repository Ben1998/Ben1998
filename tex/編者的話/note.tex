
\begin{quotation}
《少年阿賓》是由網名為Ben的網友,於1998年8月23日起在「熱站網路世界」陸續發表的網路小說,其時空背景為1980年代生活在台灣的年輕男女,在台灣網站的成人文學版受矚目。

有評論者稱此小說「整個故事時間跨度長、人物多、恩怨糾纏,為早期情色文學少有的長篇鉅著。」

小說全系列共分三部:

《鄰居的愛》:最先發表,共四章;描寫與鈺慧結婚後的家庭生活,為此系列故事的最後部份。

《少年阿賓》:篇幅最多,共六十七章;以阿賓的大學生時代為主軸,章章有不同的故事情節與人物角色。

《Three Amigo》:共三章,敘述職場上的感情故事。

文中露骨地描寫了阿賓從大學時期的愛慾故事,包括與女友鈺慧的感情、和學姊之間的事件、周圍人物的性慾糾葛;整體文章風格充滿健康活潑氣息,在肉慾之外亦令人感受到青春的無限魅力。

\end{quotation}

本次收錄的《少年阿賓》《鄰居的愛》《Three Amigo》主要來自於sexstorychinese.com\footnote{http://www.sexstorychinese.com/ben/bensex.htm}, 參考 xbookcn.com\footnote{http://www.xbookcn.com/book/abin/67.htm} 補上了《少年阿賓》第六十七章。收錄時僅作了必要的排印修正。

\chapter*{勘誤}
下文中的詞語錯誤,如在文中出現多次,僅列出最早出現的一次修改。
\begin{enumerate}
\item 補全西文空格
\item 《Three Amigo》第一章《Candy》:錯誤:「鮮細的腰際」;正確:「纖細的腰際」
\item 《Three Amigo》第三章《羚羚》:錯誤:「抽慉」;正確:「抽搐」
\item 《少年阿賓》第二章《學姐》:錯誤:「教輕鬆的短T恤」;正確:「較輕鬆的短T恤」
\item 《少年阿賓》第二章《學姐》:錯誤:「心理甜甜的」;正確:「心裡甜甜的」
\item 《少年阿賓》第二章《學姐》:錯誤:「站抖不已」;正確:「顫抖不已」
\item 《少年阿賓》第二章《學姐》:錯誤:「嚐試」;正確:「嘗試」
\item 《少年阿賓》第二章《學姐》:錯誤:「倆個」;正確:「兩個」
\item 《少年阿賓》第二章《學姐》:錯誤:「倆人」;正確:「兩人」
\item 《少年阿賓》第五章《圖書館》:錯誤:「筱雲」;正確:「鈺慧」\footnote{只有本篇出現了「筱雲」一角,查看不同的《少年阿賓》版本可以發現第五篇有的部分替換成了「鈺慧」,但仍保留個別「筱雲」字眼。編輯這裡還有2006年從文心閣復制下來的少年阿賓全本,在這個版本中已經全部替換成了「鈺慧」。興許是最初發表時使用了「筱雲」的名字,後面改成了「鈺慧」,但沒有找到依據。}
\item 《少年阿賓》第五章《圖書館》:錯誤:「檢拾」;正確:「撿拾」
\item 《少年阿賓》第五章《圖書館》:錯誤:「原來是她個騷底女人」;正確:「原來她是個騷底女人」
\item 《少年阿賓》第五章《圖書館》:錯誤:「心理老不願意」;正確:「心裡老不願意」
\item《少年阿賓》第五章《圖書館》:錯誤:「泡製」;正確:「炮製」
\item 《少年阿賓》第六章《逛街》:錯誤:「阿賓卻找不著淑華了」;正确:「阿輝卻找不著淑華了」
\item 《少年阿賓》第六章《逛街》:錯誤:「搾光他最後的餘力」;正确:「榨光他最後的餘力」
\item 《少年阿賓》第七章《打工》:錯誤:「一個PatrTime的工作」;正確:「一個Part-time的工作」
\item 《少年阿賓》第七章《打工》:錯誤:「萬分佩服」;正確:「萬分佩服。」
\item 《少年阿賓》第九章《蓮蓮》:錯誤:「抽蓄」;正確:「抽搐」
\item 《少年阿賓》第十章《寒假開始》:錯誤:「低頭幫她著奶子」;正確:「低頭幫她吸著奶子」\footnote{此處有缺字,文心閣轉載本作「低頭幫她吸奶子」,結合「低頭」語境,這裡取「低頭幫她吸著奶子」。}
\item 《少年阿賓》第十一章《表妹孟卉》:錯誤:「正客廳在抹地板」;正確:「正在客廳抹地板」
\item 《少年阿賓》第十二章《新母女關係》:錯誤:「摩蹭」;正確:「磨蹭」
\item 《少年阿賓》第十二章《新母女關係》:錯誤:「鞠躬盡悴」;正確:「鞠躬盡瘁」
\item 《少年阿賓》第十二章《新母女關係》:錯誤:「鈺惠」;正確:「鈺慧」
\item 《少年阿賓》第十三章《鑰匙遊戲》:錯誤:「鏗緣一面」;正確:「緣慳一面」
\item 《少年阿賓》第十五章《浴室春嬉》:錯誤:「心理產生異樣的快感」;正確:「心裡產生異樣的快感」
\item 《少年阿賓》第十五章《浴室春嬉》:錯誤:「髮稍」;正確:「髮梢」
\item 《少年阿賓》第十六章《美人計》:錯誤:「嫻慧」;正確:「賢慧」
\item 《少年阿賓》第二十章《萬里桐》:錯誤:「他他指的是淑華」;正確:「他指的是淑華」
\item 《少年阿賓》第二十章《萬里桐》:錯誤:「跑向過」;正確:「跑向」
\item 《少年阿賓》第廿一章《仲夏夜之夢》:錯誤:「二條」;正確:「兩條」
\item 《少年阿賓》第廿二章《同學會》:錯誤:「憶如卻高朓健美又肉感」;正確:「憶如卻高挑健美又肉感」
\item 《少年阿賓》第廿五章《媽媽的女兒》:錯誤:「晃攸攸」;正確:「晃悠悠」
\item 《少年阿賓》第廿五章《媽媽的女兒》:錯誤:「這學期的操性你及格了」;正確:「這學期的操行你及格了」
\item 《少年阿賓》第廿六章《A=A+1》:錯誤:「徹退」;正確:「撤退」
\item 《少年阿賓》第廿七章《參加婚禮》:錯誤:「規規舉舉」;正確:「規規矩矩」
\item 《少年阿賓》第廿七章《參加婚禮》:錯誤:「九宵雲外」;正確:「九霄雲外」
\item 《少年阿賓》第廿九章《奇妙婦人心》:錯誤:「麗香被幹得發昏第十一」;正確:「麗香被幹得發昏」
\item 《少年阿賓》第卅一章《意外》:錯誤:「敏妮」;正確:「敏霓」
\item 《少年阿賓》第卅一章《意外》:錯誤:「鬆馳」;正確:「鬆弛」
\item 《少年阿賓》第卅一章《意外》:錯誤:「急燥」;正確:「急躁」
\item 《少年阿賓》第卅一章《意外》:錯誤:「坐在與浴缸邊」;正確:「坐在浴缸邊」
\item 《少年阿賓》第卅一章《意外》:錯誤:「身份証」;正確:「身分證」
\item 《少年阿賓》第卅二章《機車行》:錯誤:「媚力」;正確:「魅力」
\item 《少年阿賓》第卅三章《多事KTV》:錯誤:「心理陣陣悸動」;正確:「心裡陣陣悸動」
\item 《少年阿賓》第卅三章《多事KTV》:錯誤:「現再」;正確:「現在」
\item 《少年阿賓》第卅三章《多事KTV》:錯誤:「柔胰」;正確:「柔荑」
\item 《少年阿賓》第卅三章《多事KTV》:錯誤:「雲宵飛車」;正確:「雲霄飛車」
\item 《少年阿賓》第卅三章《多事KTV》:錯誤:「替剔開」;正確:「剔開」
\item 《少年阿賓》第卅三章《多事KTV》:錯誤:「叨念」;正確:「叨念」
\item 《少年阿賓》第卅四章《成長》:錯誤:「紮實的校屁股」;正確:「紮實的小屁股」
\item 《少年阿賓》第卅四章《成長》:錯誤:「夢卉」;正確:「孟卉」
\item 《少年阿賓》第卅四章《成長》:錯誤:「阿賓像往常一樣的來鈺慧」;正確:「阿賓像往常一樣的來接鈺慧」
\item 《少年阿賓》第卅五章《溫泉》:錯誤:「蘊釀」;正確:「醞釀」
\item 《少年阿賓》第卅五章《溫泉》:錯誤:「淫盪」;正確:「淫蕩」
\item 《少年阿賓》第卅七章《訂情雨》:錯誤:「善罷干休」;正確:「善罷甘休」
\item 《少年阿賓》第卅八章《月夜眠》:錯誤:「再接再勵」;正確:「再接再厲」
\item 《少年阿賓》第卅八章《月夜眠》:錯誤:「一偶」;正確:「一隅」
\item 《少年阿賓》第卅九章《看日出》:錯誤:「孤癖」;正確:「孤僻」
\item 《少年阿賓》第卅九章《看日出》:錯誤:「博鬥」;正確:「搏鬥」
\item 《少年阿賓》第四十章《新堀江》:錯誤:「造形」;正確:「造型」
\item 《少年阿賓》第四十三章《習泳》: 錯誤:「她心理咕噥著」;正確:「她心裡咕噥著」
\item 《少年阿賓》第四十三章《習泳》: 錯誤:「隆翹起」;正確:「翹起」
\item 《少年阿賓》第四十三章《習泳》: 錯誤:「味口」;正確:「胃口」
\item 《少年阿賓》第四十四章《邊緣》: 錯誤:「她的意思」;正確:「他的意思」
\item 《少年阿賓》第四十五章《一日之計在於晨》:錯誤:「開完笑」;正確:「開玩笑」
\item 《少年阿賓》第四十五章《一日之計在於晨》:錯誤:「得笑得很開心」;正確:「笑得很開心」
\item 《少年阿賓》第四十五章《一日之計在於晨》:錯誤:「軟棉棉」;正確:「軟綿綿」
\item《少年阿賓》第四十八章《澎湖灣》:錯誤:「以經」;正確:「已經」
\item《少年阿賓》第四十八章《澎湖灣》:錯誤:「橋得慌」;正確:「瞧得慌」
\item《少年阿賓》第四十八章《澎湖灣》:錯誤:「綿被」;正確:「棉被」
\item《少年阿賓》第五十三章《暗渡》:錯誤:「型狀」;正確:「形狀」
\item 《少年阿賓》第五十五章《相逢何必曾相識》: 錯誤:「心理產生一些詭奇的快感」;正確:「心裡產生一些詭奇的快感」
\item 《少年阿賓》第六十章《脫殼》:錯誤:「搬妥Bye-bey了」;正確:「搬妥Bye-bye了」
\item 《少年阿賓》第六十四章《參差荇菜》:錯誤:「虎視耽耽」;正確:「虎視眈眈」
\item 《少年阿賓》第六十七章《Walk Through》:錯誤:「開門緝盜」;正確:「開門揖盜」
\end{enumerate}
